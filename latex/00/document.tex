\documentclass[11pt,twoside,a4paper,pagesize]{report}
\usepackage{pslatex,palatino,avant,,color}
%\usepackage{graphicx}
\usepackage[demo]{graphicx}
\usepackage{nextpage}
\usepackage{afterpage}
\usepackage[margin=2cm]{geometry}
\usepackage{comment}
\usepackage[table,xcdraw]{xcolor}
\usepackage{fancyvrb}
\usepackage{listings}
\usepackage[htt]{hyphenat}
\usepackage[font=itshape]{quoting}
\usepackage[absolute]{textpos}
\usepackage{typearea}
\usepackage[hyphens]{url}
\PassOptionsToPackage{hyphens}{url}
\usepackage{paralist}
\usepackage[normalem]{ulem}
\begin{comment}
    \usepackage{hyperref}
\end{comment}
\setlength{\parskip}{.35\baselineskip}

\expandafter\def\expandafter\UrlBreaks\expandafter{\UrlBreaks%  save the current one
  \do\a\do\b\do\c\do\d\do\e\do\f\do\g\do\h\do\i\do\j%
  \do\k\do\l\do\m\do\n\do\o\do\p\do\q\do\r\do\s\do\t%
  \do\u\do\v\do\w\do\x\do\y\do\z\do\A\do\B\do\C\do\D%
  \do\E\do\F\do\G\do\H\do\I\do\J\do\K\do\L\do\M\do\N%
  \do\O\do\P\do\Q\do\R\do\S\do\T\do\U\do\V\do\W\do\X%
  \do\Y\do\Z}

\begin{comment}
\textfloatsep — distance between floats on the top or the bottom and the text;
\floatsep — distance between two floats;
\intextsep — distance between floats inserted inside the page text (using h) and the text proper.
The command used to change them is \setlength:

\setlength{\textfloatsep}{10pt plus 1.0pt minus 2.0pt}
\end{comment}

\setlength{\intextsep}{30pt}

\usepackage{color}

\definecolor{mygreen}{rgb}{0,0.6,0}
\definecolor{mygray}{rgb}{0.5,0.5,0.5}
\definecolor{mymauve}{rgb}{0.58,0,0.82}

\begin{comment}
\lstset{numbers=left,
  inputencoding=latin1,
  basicstyle=\fontsize{9}{12}\selectfont\ttfamily,
  basewidth  = {0.5em,0.5em},
  keywordstyle=\color{blue},
  commentstyle=\color{mygreen},
  breaklines=true, 
  showtabs=false,
  showstringspaces=false,
  numberstyle=\tiny\color{mygray}
}
\end{comment}

\lstdefinestyle{default-style}{
  inputencoding=latin1,
  basicstyle=\fontsize{9}{12}\selectfont\ttfamily,
  basewidth  = {0.5em,0.5em},
  keywordstyle=\color{blue},
  commentstyle=\color{mygreen},
  breaklines=true, 
  showtabs=false,
  showstringspaces=false,
  numbers             =left, 
  stepnumber          =   1,  
  numbersep           =   10pt,
  numberstyle=\tiny\color{mygray}
}

\lstdefinestyle{small-style}{
  inputencoding=latin1,
  basicstyle=\fontsize{8}{10}\selectfont\ttfamily,
  basewidth  = {0.5em,0.5em},
  keywordstyle=\color{blue},
  commentstyle=\color{mygreen},
  breaklines=true, 
  showtabs=false,
  showstringspaces=false,
  numbers             =   left, 
  stepnumber          =   1,  
  numbersep           =   10pt,
  numberstyle=\tiny\color{mygray}
}

\lstset{style=default-style}

\begin{document}
\title{\color{blue}Document Title}
\author{Menelaus Perdikeas\\ Company SA}
\date{June 2014}
\maketitle

\tableofcontents
\listoffigures
\lstlistoflistings

\renewcommand{\abstractname}{Executive Summary}
\begin{abstract}
...
\end{abstract}

\chapter{Introduction}
As can be read in the \textit{Ten Commandments} \cite{ref:ten-commandments} ..

\begin{quoting}
fancy quote
\end{quoting}

\uline{You can underline too}.

Here's how to use inline enumerations:
\begin{inparaenum}[i)]
\item first thing
\item second thing, and
\item third thing
\end{inparaenum}.

Figure \ref{fig:black-knight} shows a typical scene from the block-buster
film ``The Black Knight''.

% comments

\begin{comment}
    another way to write comments
\end{comment}

\begin{comment}
\begin{figure}[htbp]
    \centering
    \includegraphics[width=1.0\textwidth]{BlackKnight.png}
    \caption{The Black Knight}
    \label{fig:black-knight}
\end{figure}
\end{comment}


\afterpage{ % Insert after the current page
\clearpage

\KOMAoptions{paper=a3}
\addtolength{\hoffset}{-2.0cm}
\recalctypearea
\begin{figure}[htbp]
    \centering
    %\includegraphics[width=1.25\textwidth]{vo-business-schema.png}
    \includegraphics[width=1.35\textwidth]{}
    \caption{\texttt{vo\ business} schema tables}
    \label{fig:vo-business-schema}
\end{figure}

\clearpage
\KOMAoptions{paper=A4,pagesize}
\recalctypearea
}
\restoregeometry




The pseudocode of listing \ref{lst:movie-instructions} shows how to
make a block-buster.

\lstset{language=Java}
\begin{lstlisting}[float, frame=single, caption=How to make a block-buster, label=lst:movie-instructions]

while (enoughMoneyAtTheBank()) {
    buyBookRights();
    contractFemaleStar();
    produceMovie();
}
\end{lstlisting}


\clearpage

The document is organized as follows\footnote{Footnote here.}:
\begin{itemize}
\item Chapter ... 
\end{itemize}

\afterpage{ % Insert after the current page
\clearpage

\KOMAoptions{paper=a3,paper=landscape} %requires \usepackage{typearea} in the preamble and 'pagesize' as an attribute to the document class (http://tex.stackexchange.com/q/179714/52217)
\recalctypearea
A3 stuff in landscape goes here ...
\clearpage
\KOMAoptions{paper=A4,pagesize}
\recalctypearea
}

\clearpage
%\restoregeometry % NB: in one document I had to explicitly invoke the \restoregeometry command to restore geometry, apparently it is not needed in this archetype.
Back on A3\footnote{Footnotes and normal page geometry is restored.}.



\renewcommand{\bibname}{References}

\begin{thebibliography}{1}
    \bibitem{ref:ten-commandments}God {\em The Ten Commandments}, carved in stone, 2000BC, Mount Sinai Publishing
\end{thebibliography}

\end{document}

